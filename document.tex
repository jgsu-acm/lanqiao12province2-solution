\documentclass{pptt}

\title{第十二届蓝桥杯省赛第二场题解}
\author{ACM算法与微应用开发实验室 \and AgOH}
\date{2022 年 3 月 12 日}

\begin{document}
\maketitle

\section{结果填空题}

\begin{frame}{求余}{题面}
    在 C/C++/Java/Python 等语言中,使用 $\%$ 表示求余,请问 $2021 \% 20$ 的值是多少?
\end{frame}

\begin{frame}{求余}{题解}
    送分题
    
    答案:$1$
\end{frame}

\begin{frame}{双阶乘}{题面}
    一个正整数的双阶乘,表示不超过这个正整数且与它有相同奇偶性的所有正整数乘积。$n$ 的双阶乘用 $n!!$ 表示。

    例如:

    \begin{itemize}
        \item $3!! = 3 \times 1 = 3$。
        \item $8!! = 8 \times 6 \times 4 \times 2 = 384$。
        \item $11!! = 11 \times 9 \times 7 \times 5 \times 3 \times 1 = 10395$。
    \end{itemize}

    请问,$2021!!$ 的最后 $5$ 位(这里指十进制位)是多少?
\end{frame}

\begin{frame}{双阶乘}{题解}
    发现我们只需要维护最后 $5$ 位即可,于是一个 for 搞定,边乘边对 ${10}^5$ 取模

    答案:$59375$
\end{frame}

\begin{frame}{格点}{题面}
    如果一个点 $(x,y)$ 的两维坐标都是整数,即 $x \in \mathbb{Z}$ 且 $y \in \mathbb{Z}$,则称这个点为一个格点。

    如果一个点 $(x,y)$ 的两维坐标都是正数,即 $x>0$ 且 $y>0$,则称这个点在第一象限。

    请问在第一象限的格点中,有多少个点 $(x,y)$ 的两维坐标乘积不超过 $2021$,即 $x \cdot y \leq 2021$。
\end{frame}

\begin{frame}{格点}{题解}
    两重 for 遍历 $x,y$ 值然后加一个判断计数即可

    答案:$15698$
\end{frame}

\begin{frame}{整数分解}{题面}
    将 $3$ 分解成两个正整数的和,有两种分解方法,分别是 $3=1+2$ 和 $3=2+1$。注意顺序不同算不同的方法。

    将 $5$ 分解成三个正整数的和,有 $6$ 种分解方法,它们是 $1+1+3 = 1+2+2 = 1+3+1 = 2+1+2 = 2+2+1 = 3+1+1$。

    请问,将 $2021$ 分解成五个正整数的和,有多少种分解方法?
\end{frame}

\begin{frame}{整数分解}{题解}
    注意到所求为方案数,求方案数常见两种方法:组合数学或动态规划

    本题两种方法均可

    \begin{enumerate}
        \item 组合数学

              显然题目即为在 $2021$ 个 $1$ 的 $2020$ 个空隙中插入 $4$ 个隔板的方案数

              故答案为:$\binom{2020}{4}=691677274345$
        \item 动态规划
              \begin{itemize}
                  \item 状态设计:$dp[i][j]$ 表示 $i$ 分成 $j$ 个正整数之和的方案数
                  \item 初始状态:$dp[i][1] = 1$
                  \item 转移方程:$dp[i][j] = \sum_{k=1}^{i-1} dp[k][j-1]$
                  \item 所求结果:$dp[2021][5] = 691677274345$
              \end{itemize}
    \end{enumerate}

    答案:$691677274345$
\end{frame}

\begin{frame}{城邦}{题面}
    小蓝国是一个水上王国,有 $2021$ 个城邦,依次编号 $1$ 到 $2021$。在任意两个城邦之间,都有一座桥直接连接。

    为了庆祝小蓝国的传统节日,小蓝国政府准备将一部分桥装饰起来。

    对于编号为 $a$ 和 $b$ 的两个城邦,它们之间的桥如果要装饰起来,需要的费用如下计算:找到 $a$ 和 $b$ 在十进制下所有不同的数位,将数位上的数字求和。

    例如,编号为 $2021$ 和 $922$ 两个城邦之间,千位、百位和个位都不同,将这些数位上的数字加起来是 $(2+0+1)+(0+9+2)=14$。注意 $922$ 没有千位,千位看成 $0$。

    为了节约开支,小蓝国政府准备只装饰 $2020$ 座桥,并且要保证从任意一个城邦到任意另一个城邦之间可以完全只通过装饰的桥到达。

    请问,小蓝国政府至少要花多少费用才能完成装饰。
\end{frame}

\begin{frame}{城邦}{题解}
    显然最小生成树裸题,下一道

    答案:$4046$
\end{frame}

\begin{frame}{游戏}{题面}
    小蓝闲着无聊开始自己和自己做游戏。

    首先规定一个正整数 $n$。

    他首先在纸上写下一个 $1$ 到 $n$ 之间的数。在之后的每一步,小蓝都可以选择上次写的数的一个约数(不能选上一个写过的数),写在纸上。直到最终小蓝写下 $1$。

    小蓝可能有多种游戏的方案。

    例如,当 $n=6$ 时,小蓝有 $9$ 种方案:$(1), (2,1), (3,1), (4,1), (4,2,1), (5,1),(6,1), (6,2,1), (6,3,1)$。

    请问,当 $n=20210509$ 时有多少种方案?
\end{frame}

\begin{frame}{游戏}{题解}
    又是求方案数,发现这回不能用组合数学了,于是思路转向 DP

    \begin{itemize}
        \item 状态设计:$dp[i]$ 表示 $n=i$ 时的方案数
        \item 初始状态:$dp[1] = 1$
        \item 转移方程:$dp[i] = \sum_{d|i} dp[d]$
        \item 所求结果:$dp[n]$
    \end{itemize}

    发现递推貌似不太行,找一个数的约数所花时间显然接受不了。于是我们采用刷表法,用前向状态去更新后继状态。过程类似埃氏筛

    注意过程中会出现很大的数,请用 Python 来写

    答案:$1352184317599$
\end{frame}

\section{程序设计题}

\begin{frame}{特殊年份}{题面}
    今年是 $2021$ 年,$2021$ 这个数字非常特殊,它的千位和十位相等,个位比百位大 $1$,我们称满足这样条件的年份为特殊年份。

    输入 $5$ 个年份,请计算这里面有多少个特殊年份。
\end{frame}

\begin{frame}{特殊年份}{题解}
\end{frame}

\begin{frame}{小平方}{题面}
    小蓝发现,对于一个正整数 $n$ 和一个小于 $n$ 的正整数 $v$,将 $v$ 平方后对 $n$ 取余可能小于 $n$ 的一半,也可能大于等于 $n$ 的一半。

    请问,在 $1$ 到 $n-1$ 中,有多少个数平方后除以 $n$ 的余数小于 $n$ 的一半。
\end{frame}

\begin{frame}{小平方}{题解}
\end{frame}

\begin{frame}{完全平方数}{题面}
    一个整数 $a$ 是一个完全平方数,是指它是某一个整数的平方,即存在一个整数 $b$,使得 $a=b^2$。

    给定一个正整数 $n$,请找到最小的正整数 $x$,使得它们的乘积是一个完全平方数。
\end{frame}

\begin{frame}{完全平方数}{题解}
\end{frame}

\begin{frame}{负载均衡}{题面}
    有 $n$ 台计算机,第 $i$ 台计算机的运算能力为 $v_i$。

    有一系列的任务被指派到各个计算机上,第 $i$ 个任务在 $a_i$ 时刻分配,指定计算机编号为 $b_i$,耗时为 $c_i$ 且算力消耗为 $d_i$。如果此任务成功分配,将立刻开始运行,期间持续占用 $b_i$ 号计算机 $d_i$ 的算力,持续 $c_i$ 秒。

    对于每次任务分配,如果计算机剩余的运算能力不足则输出 $-1$,并取消这次分配,否则输出分配完这个任务后这台计算机的剩余运算能力。
\end{frame}

\begin{frame}{负载均衡}{题解}
\end{frame}

\begin{frame}{国际象棋}{题面}
    众所周知,“八皇后”问题是求解在国际象棋棋盘上摆放 $8$ 个皇后,使得两两之间互不攻击的方案数。已经学习了很多算法的小蓝觉得“八皇后”问题太简单了,意犹未尽。作为一个国际象棋迷,他想研究在 $N \times M$ 的棋盘上,摆放 $K$ 个马,使得两两之间互不攻击有多少种摆放方案。由于方案数可能很大,只需计算答案除以 ${10}^9+7$ 的余数。
\end{frame}

\begin{frame}{国际象棋}{题解}
\end{frame}

\begin{frame}{完美序列}{题面}
    一个序列中取出一些元素按照原来的顺序排列成新的序列称为该序列的一个子序列。子序列的价值为子序列中所有元素的和。

    如果一个序列是单调递减的,而且除了第一个数以外的任何一个数都是上一个数的因数,则称这个序列为一个完美序列。

    一个序列中的一个子序列如果是完美序列,则称为该序列的一个完美子序列。一个序列的最长完美子序列长度,称为该序列的完美长度。

    给定正整数 $n$,$1$ 至 $n$ 的所有排列的完美长度的最大值,称为 $n$ 阶最大完美长度。

    给定正整数 $n$,请求出 $1$ 至 $n$ 的所有排列中长度正好为 $n$ 阶最大完美长度的所有完美子序列的价值的和。
\end{frame}

\begin{frame}{完美序列}{题解}
\end{frame}

\end{document}
