\documentclass{pptt}
\usepackage{xcolor}
\usepackage{colortbl}
\usepackage{tikz}
\usetikzlibrary{tikzmark}

\newcommand{\colormark}[3]{\tikzmarknode{#1}{\colorbox{#2}{#3}}}
\newcolumntype{e}{>{\columncolor{red!20}}c}
\newcolumntype{u}{>{\columncolor{blue!20}}c}

\title{第十二届蓝桥杯省赛第二场 C/C++ 组题解}
\author{ACM算法与微应用开发实验室 \and AgOH}
\date{2022 年 3 月 12 日}

\begin{document}
\maketitle

\section{求余}

\begin{frame}{题面}
    在 C/C++/Java/Python 等语言中,使用 $\%$ 表示求余,请问 $2021 \% 20$ 的值是多少?
\end{frame}

\begin{frame}{题解}
    送分题

    答案:$1$
\end{frame}

\section{双阶乘}

\begin{frame}{题面}
    一个正整数的双阶乘,表示不超过这个正整数且与它有相同奇偶性的所有正整数乘积。$n$ 的双阶乘用 $n!!$ 表示。

    例如:

    \begin{itemize}
        \item $3!! = 3 \times 1 = 3$。
        \item $8!! = 8 \times 6 \times 4 \times 2 = 384$。
        \item $11!! = 11 \times 9 \times 7 \times 5 \times 3 \times 1 = 10395$。
    \end{itemize}

    请问,$2021!!$ 的最后 $5$ 位(这里指十进制位)是多少?
\end{frame}

\begin{frame}{题解}
    只需要维护最后 $5$ 位即可,于是一个 \texttt{for} 搞定,边乘边对 ${10}^5$ 取模

    答案:$59375$
\end{frame}

\section{格点}

\begin{frame}{题面}
    如果一个点 $(x,y)$ 的两维坐标都是整数,即 $x \in \mathbb{Z}$ 且 $y \in \mathbb{Z}$,则称这个点为一个格点。

    如果一个点 $(x,y)$ 的两维坐标都是正数,即 $x>0$ 且 $y>0$,则称这个点在第一象限。

    请问在第一象限的格点中,有多少个点 $(x,y)$ 的两维坐标乘积不超过 $2021$,即 $x \cdot y \leq 2021$。
\end{frame}

\begin{frame}{题解}
    两重 \texttt{for} 遍历 $x,y$ 值然后加一个判断计数即可

    答案:$15698$
\end{frame}

\section{整数分解}

\begin{frame}{题面}
    将 $3$ 分解成两个正整数的和,有两种分解方法,分别是 $3=1+2$ 和 $3=2+1$。注意顺序不同算不同的方法。

    将 $5$ 分解成三个正整数的和,有 $6$ 种分解方法,它们是:
    
    $$(1+1+3) = (1+2+2) = (1+3+1) = (2+1+2) = (2+2+1) = (3+1+1)$$

    请问,将 $2021$ 分解成五个正整数的和,有多少种分解方法?
\end{frame}

\begin{frame}{题解}
    注意到所求为方案数,求方案数常见两种方法:组合数学或动态规划

    本题两种方法均可

    \begin{enumerate}
        \item 组合数学

              显然题目即为在 $2021$ 个 $1$ 的 $2020$ 个空隙中插入 $4$ 个隔板的方案数

              故答案为:$\binom{2020}{4}=691677274345$
        \item 动态规划
              \begin{itemize}
                  \item 状态设计:$dp[i][j]$ 表示 $i$ 分成 $j$ 个正整数之和的方案数
                  \item 初始状态:$dp[i][1] = 1$
                  \item 转移方程:$$dp[i][j] = \sum_{k=1}^{i-1} dp[k][j-1]$$
                  \item 所求结果:$dp[2021][5] = 691677274345$
              \end{itemize}
    \end{enumerate}

    答案:$691677274345$
\end{frame}

\section{城邦}

\begin{frame}{题面}
    小蓝国是一个水上王国,有 $2021$ 个城邦,依次编号 $1$ 到 $2021$。在任意两个城邦之间,都有一座桥直接连接。

    为了庆祝小蓝国的传统节日,小蓝国政府准备将一部分桥装饰起来。

    对于编号为 $a$ 和 $b$ 的两个城邦,它们之间的桥如果要装饰起来,需要的费用如下计算:找到 $a$ 和 $b$ 在十进制下所有不同的数位,将数位上的数字求和。

    例如,编号为 $2021$ 和 $922$ 两个城邦之间,千位、百位和个位都不同,将这些数位上的数字加起来是 $(2+0+1)+(0+9+2)=14$。注意 $922$ 没有千位,千位看成 $0$。

    为了节约开支,小蓝国政府准备只装饰 $2020$ 座桥,并且要保证从任意一个城邦到任意另一个城邦之间可以完全只通过装饰的桥到达。

    请问,小蓝国政府至少要花多少费用才能完成装饰。
\end{frame}

\begin{frame}{题解}
    显然最小生成树裸题,下一道

    答案:$4046$
\end{frame}

\section{游戏}

\begin{frame}{题面}
    小蓝闲着无聊开始自己和自己做游戏。

    首先规定一个正整数 $n$。

    他首先在纸上写下一个 $1$ 到 $n$ 之间的数。在之后的每一步,小蓝都可以选择上次写的数的一个约数(不能选上一个写过的数),写在纸上。直到最终小蓝写下 $1$。

    小蓝可能有多种游戏的方案。

    例如,当 $n=6$ 时,小蓝有 $9$ 种方案:
    
    $$(1), (2,1), (3,1), (4,1), (4,2,1), (5,1), (6,1), (6,2,1), (6,3,1)$$

    请问,当 $n=20210509$ 时有多少种方案?
\end{frame}

\begin{frame}{题解}
    又是求方案数,发现这回不能用组合数学了,于是思路转向 DP

    \begin{itemize}
        \item 状态设计:$dp[i]$ 表示从 $i$ 开始写的方案数
        \item 初始状态:$dp[1] = 1$
        \item 转移方程:$$dp[i] = \sum_{d|i} dp[d]$$
        \item 所求结果:$$\sum_{i=1}^{n} dp[i]$$
    \end{itemize}

    发现直接递推的话,对于每个数都需要 $O(\sqrt n)$ 找约数,时间复杂度是 $O(n \sqrt n)$ 的,有些慢。于是我们采用刷表法优化,用前向状态去更新后继状态,即用每个数的 DP 值去更新其倍数的 DP 值。时间复杂度 $O(n \log n)$

    答案:$1352184317599$
\end{frame}

\section{特殊年份}

\begin{frame}{题面}
    今年是 $2021$ 年,$2021$ 这个数字非常特殊,它的千位和十位相等,个位比百位大 $1$,我们称满足这样条件的年份为特殊年份。

    输入 $5$ 个年份,请计算这里面有多少个特殊年份。
\end{frame}

\begin{frame}{题解}
    模拟即可

    使用 \texttt{std::string} 可以非常简单地写出程序
\end{frame}

\section{小平方}

\begin{frame}{题面}
    小蓝发现,对于一个正整数 $n$ 和一个小于 $n$ 的正整数 $v$,将 $v$ 平方后对 $n$ 取余可能小于 $n$ 的一半,也可能大于等于 $n$ 的一半。

    请问,在 $1$ 到 $n-1$ 中,有多少个数平方后除以 $n$ 的余数小于 $n$ 的一半。
\end{frame}

\begin{frame}{题解}
    模拟……开个 \texttt{for} 循环即可
\end{frame}

\section{完全平方数}

\begin{frame}{题面}
    一个整数 $a$ 是一个完全平方数,是指它是某一个整数的平方,即存在一个整数 $b$,使得 $a=b^2$。

    给定一个正整数 $n$,请找到最小的正整数 $x$,使得它们的乘积是一个完全平方数。
\end{frame}

\begin{frame}{题解}
    设 $A$ 为完全平方数,有:$A=x \times x$

    设 $x$ 的标准质因数分解式为:

    $$x={p_1}^{\alpha_1}{p_2}^{\alpha_2} \cdots {p_s}^{\alpha_s}$$

    则 $A$ 的标准质因数分解式为:

    $$A={p_1}^{2\alpha_1}{p_2}^{2\alpha_2} \cdots {p_s}^{2\alpha_s}$$

    即 $A$ 的所有质因数都一定出现偶数次,故我们只需要对 $n$ 进行质因数分解,然后将所有出现了奇数次方的质因数乘在一起即为答案(将奇数补成偶数)。时间复杂度:$O(\sqrt n)$
\end{frame}

\section{负载均衡}

\begin{frame}{题面}
    有 $n$ 台计算机,第 $i$ 台计算机的运算能力为 $v_i$。

    有一系列的任务被指派到各个计算机上,第 $i$ 个任务在 $a_i$ 时刻分配,指定计算机编号为 $b_i$,耗时为 $c_i$ 且算力消耗为 $d_i$。如果此任务成功分配,将立刻开始运行,期间持续占用 $b_i$ 号计算机 $d_i$ 的算力,持续 $c_i$ 秒。

    对于每次任务分配,如果计算机剩余的运算能力不足则输出 $-1$,并取消这次分配,否则输出分配完这个任务后这台计算机的剩余运算能力。
\end{frame}

\begin{frame}{题解}
    可以把分配任务想象成向计算机借了一些算力,时间到了后再还回去

    于是我们开一个堆来维护要还回去的算力,对于每次操作先把时间到了的需要还回去的算力还回去,再判断此次能否借来算力,若能借来算力,把计算机的算力扣除了借走的算力后,在堆中打一个欠条。重复此流程即可,时间复杂度:$O(n \log n)$
\end{frame}

\section{国际象棋}

\begin{frame}{题面}
    众所周知,“八皇后”问题是求解在国际象棋棋盘上摆放 $8$ 个皇后,使得两两之间互不攻击的方案数。已经学习了很多算法的小蓝觉得“八皇后”问题太简单了,意犹未尽。作为一个国际象棋迷,他想研究在 $N \times M$ 的棋盘上,摆放 $K$ 个马,使得两两之间互不攻击有多少种摆放方案。由于方案数可能很大,只需计算答案除以 ${10}^9+7$ 的余数。
\end{frame}

\begin{frame}{题解}
    发现此题即 \href{https://www.luogu.com.cn/problem/P1896}{\color{blue} [SCOI2005] 互不侵犯} 的加强版。把王换成马,即会跟上两行有关系罢了

    状压 DP,把每行棋子放法用 01 串来表示,放了马为 1,没放为 0:

    \begin{itemize}
        \item 状态设计:$dp[i][p][q][j]$ 代表当前放到第 $i$ 行,放法为 $p$,上一行放法为 $q$,已经放了 $j$ 个马的方案数
        \item 初始状态:$dp[1][p][0][popcount(p)] = 1,p \in [0, 2^n-1]$ {\scriptsize (同一行的马不会互相攻击所以可以随便放)}
        \item 转移方程:$$dp[i][p][q][j] = \sum_{a} dp[i-1][q][a][j-popcount(p)],j \in [popcount(p), k],(q,p) \And (a,p) \text{不冲突}$$
        \item 所求结果:$$\sum_{p=0}^{2^n-1} \sum_{q=0}^{2^n-1} dp[m][p][q][k]$$
    \end{itemize}

    注意数据范围 $n<=6,~m<=100$,即 $2^m >> 2^n$,故我们把 $m$ 当做行来处理可以极大地缩小空间时间复杂度,最终时间复杂度:$O(m \times 2^n \times 2^n \times 2^n \times k) = O(8^nmk)$
\end{frame}

\section{完美序列}

\begin{frame}{题面}
    一个序列中取出一些元素按照原来的顺序排列成新的序列称为该序列的一个子序列。子序列的价值为子序列中所有元素的和。

    如果一个序列是单调递减的,而且除了第一个数以外的任何一个数都是上一个数的因数,则称这个序列为一个完美序列。

    一个序列中的一个子序列如果是完美序列,则称为该序列的一个完美子序列。一个序列的“最长完美子序列”长度,称为该序列的完美长度。

    给定正整数 $n$,$1$ 至 $n$ 的所有排列的完美长度的最大值,称为 $n$ 阶最大完美长度。

    给定正整数 $n$,请求出 $1$ 至 $n$ 的所有排列中长度正好为 $n$ 阶最大完美长度的所有完美子序列的价值的和。
\end{frame}

\begin{frame}{题解}
    \textbf{鸣谢:感谢 Tifa 大佬提供的主要思路}

    \vspace{1cm}
    读完题目,我们可以轻松获取如下三个结论(以下简称“$1$ 至 $n$ 的所有排列中长度正好为 $n$ 阶最大完美长度的最长完美子序列”为“最长完美子序列”:

    \begin{enumerate}
        \item $n$ 阶最大完美长度为 $\lfloor \log_2 n \rfloor + 1$
        \item 每个“最长完美子序列”一定以 $1$ 为结尾
        \item 每个“最长完美子序列”中一定只存在前一个数是后一个数的 $2$ 倍或 $3$ 倍,且 $3$ 倍只出现一次
    \end{enumerate}
\end{frame}

\begin{frame}
    第一个结论是显然的,因为下降最慢的完美序列为等比数列 $2^k,2^{k-1},\cdots,1$,故一个最大值为 $n$ 的完美序列的最大长度为 $\lfloor \log_2 n \rfloor + 1$

    第二个结论采取反证法:若存在一个“最长完美子序列” $P$ 不以 $1$ 为结尾,那么一定存在一个完美子序列 $Q=(P,1)$ 比 $P$ 长,那么 $P$ 必然不是最长的,矛盾

    第三个结论采取反证法:首先,若出现了 $3$ 倍以上,即以长度 $2$ 出现了 $3$ 倍以上(例:$5,1$),但我们至少可以用长度 $3$ 来出现 $4$ 倍($4,2,1$),也就说说若出现了 $3$ 倍以上,那么此“最长完美子序列”必然不是最长的,矛盾;其次,若出现了两个 $3$ 倍,即以长度 $3$ 出现了 $9$ 倍(例:$9,3,1$),但我们可以用长度 $4$ 来出现 $8$ 倍($8,4,2,1$),也就是说若出现了 $9$ 倍,那么此“最长完美子序列”必然不是最长的,矛盾
\end{frame}

\begin{frame}
    有了如上三个结论,我们就可以开始考虑如何解决这个问题了。设 $n$ 阶最大完美长度为 $len=\lfloor \log_2 n \rfloor + 1$,我们分情况讨论:

    \begin{itemize} \item “最长完美子序列”中不存在 $3$ 倍,即“最长完美子序列”为:$2^{len-1} \rightarrow 1$ \end{itemize}

    此时“最长完美子序列”为等比数列 $2^{len-1},2^{len-2},\cdots,1$,由等比数列求和公式 $S_n=\cfrac{a_1-a_{n}q}{1-q}$,此情况下“最长完美子序列”的和为:

    $$\cfrac{2^{len-1}-1 \times \frac{1}{2}}{1-\frac{1}{2}}=2^{len}-1$$
\end{frame}

\begin{frame}
    \begin{itemize} \item “最长完美子序列”中存在 $3$ 倍,即“最长完美子序列”为:$3 \cdot 2^{len-2} \rightarrow 1$ \end{itemize}

    注意:只有当 $n \geq 3 \cdot 2^{len-2}$ 时才会出现此情况

    设 $S_{len}$ 为存在 $3$ 倍情况下所有长度为 $len$ 的“最长完美子序列”的和,则有:

    $$
        \begin{cases}
            \colormark{t}{orange!20}{$S_2 = 4$} \\
            \begin{aligned}
                S_{len} & = \colormark{a}{red!20}{$(len-1)(3 \cdot 2^{len-2})$} +
                \colormark{b}{gray!40}{$\sum_{i=0}^{len-2} 2^{i}$} +
                \colormark{c}{blue!20}{$S_{len-1}$}                               \\
                        & = (len-1)(3 \cdot 2^{len-2}) + 2^{len-1}-1 + S_{len-1}
            \end{aligned}
        \end{cases}
    $$

    \begin{tikzpicture}[overlay,remember picture,>=stealth,nodes={align=left,inner ysep=1pt},->]
        \path (t.west) ++ (-1,0) node[anchor=east,color=orange](pt){\small $(3,1)$};
        \draw [color=orange](pt.east) -- ([color=orange]t.west);

        \path (a.north) ++ (0,0.6) node[anchor=south west,color=red!70](pa){\scriptsize 共 $len-1$(有 $len-1$ 个位置可以放 $3$倍)个以 $3 \cdot 2^{len-2}$ 开头的};
        \draw [color=red](pa.south east) -| ([color=red]a.north);

        \path (b.north) ++ (0,0.2) node[anchor=south west,color=gray](pb){\scriptsize 长度为 $len-1$ 且不包含 $3$ 倍的};
        \draw [color=gray](pb.south east) -| ([color=gray]b.north);

        \path (c.east) ++ (0.5,0) node[anchor=west,color=blue!70](pc){\scriptsize 长度为 $len-1$ 且包含 $3$ 倍的};
        \draw [color=blue](pc.west) -- ([color=blue]c.east);
    \end{tikzpicture}

    递推预处理即可。以 $len=5$ 为例,上式的各部分如下:

    $$
        \begin{array}{euuuu}
            \rowcolor{gray!40}
            \cellcolor{red!20} 24(\div 3) & 8          & 4         & 2         & 1 \\
            24                            & 12(\div 3) & 4         & 2         & 1 \\
            24                            & 12         & 6(\div 3) & 2         & 1 \\
            24                            & 12         & 6         & 3(\div 3) & 1
        \end{array}
    $$

    当然,也可以使用此递推式的通项公式 $S_{len}=2^{len-1}(3len-4) - len + 2$ 直接算出结果
\end{frame}

\begin{frame}
    于是对于一个给定的 $n$,我们只需要判断一下是否可能出现第二种情况,根据上文所述即可获取所有“最长完美子序列”的和

    注意“最长完美子序列”是 $1$ 至 $n$ 的所有排列的子序列,也就是说每个“最长完美子序列”会出现多次,其出现次数为 $\cfrac{A_n^n}{A_{len}^{len}}=\cfrac{n!}{len!}$,在计算答案时需要乘进去

    为什么是 $\cfrac{A_n^n}{A_{len}^{len}}$ 呢?$1$ 至 $n$ 的所有排列共有 $A_n^n$ 个,而其中“最长完美子序列”的相对位置是不变的,所以除掉“最长完美子序列”的排列数 $A_{len}^{len}$

    注意到出现了有理数取余,需要预处理阶乘及其逆元 $\left[ \cfrac{n!}{len!} \equiv n! \times (len!)^{-1} \pmod p \right]$。数据规模 $n \leq {10}^6$ 不是很大,采用 $O(n + \log p)$ 或者 $O(n \log n)$ 求逆元都可以。
    
    若采用 $O(n \log n)$ 求逆元则总时间复杂度为 $O(n \log n + T)$;若采用 $O(n + \log p)$ 求逆元则总时间复杂度为 $O(n + \log p + T)$
\end{frame}

\end{document}
